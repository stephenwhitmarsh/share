
\tikzstyle{sharedFeatures} = [
        rectangle, 
        inner xsep=0cm,
        inner ysep=0.2\baselineskip,
        minimum width=2.3cm, 
        minimum height=1cm, 
        % text centered, 
        draw=black, 
        align=center,
    ]
\tikzstyle{startstop} = [
        sharedFeatures, 
        rounded corners = \corners, 
    ]
\tikzstyle{manual} = [
        sharedFeatures, 
        trapezium, 
        minimum width=1.5cm, 
        minimum height=2.3cm,     
        shape border rotate=90,
        trapezium stretches,
        trapezium left angle=90,
        trapezium right angle=80,
    ]
\tikzstyle{io} = [
        sharedFeatures,
        minimum width=2.3cm,
        minimum height=1cm,
        trapezium, 
        trapezium left angle=70, 
        trapezium right angle=110,
        trapezium stretches,
        fill=black!20,
    ]
\tikzstyle{process} = [
        sharedFeatures,
    ]
\tikzstyle{decision} = [
        sharedFeatures,
        diamond,
        minimum height=2.3cm,
        inner xsep=0cm,
        inner ysep=0\baselineskip,        
        fill=black!10,
    ]
\tikzstyle{combine} = [
        sharedFeatures,
        diamond,
        minimum height=0.7cm,
        minimum width=0.7cm,
        inner xsep=0cm,
        inner ysep=0\baselineskip,        
    ]    
\tikzstyle{empty} = [
        sharedFeatures,
        draw=none,
        fill=none,
    ] 
\tikzstyle{emptySmall} = [
        sharedFeatures,
        draw=none,
        minimum height=0.7cm,
        minimum width=0.7cm,
        inner xsep=0cm,
        inner ysep=0\baselineskip,        
    ]    

\def\corners{2.mm}
\tikzstyle{arrow} = [-{Latex[scale=3,length=3,width=2]}, rounded corners = \corners]
\tikzstyle{noarrow} = [line width=1pt]

\NewDocumentCommand{\JumpOver}{ O{arrow} m O{arrow} m }{
    % Save the line to draw as Jumping Path,
    % the line(s) to avoid as Obstacles Path,
    % and a path describing our "hoop" as Jump.
    \path[#1, spath/save=Jumping Path] #2;
    \path[#3, spath/save=Obstacles Path] #4;
    \path[spath/save=Jump]
        (0, 0) .. controls (0.5, 0) and (0.5, 1) .. 
        (1.5, 1) .. controls (2.5, 1) and (2.5, 0) .. 
        (3, 0);
    
    \tikzset{
      spath/split at intersections with={Jumping Path}{Obstacles Path},
      spath/insert gaps after components={Jumping Path}{3mm},
      spath/join components with={Jumping Path}{Jump},
    }        
    \draw[#3, spath/use=Obstacles Path];
    \draw[#1, spath/use=Jumping Path];
}

\scalebox{0.7}{
\begin{tikzpicture}[node distance=0.7cm and 0.7cm,
      line join = round,
      line width = 1pt,
      auto,
      start chain = going down,
      box/.style = {draw, rectangle, font=\huge, on chain},
      L/.style = {draw, black, rounded corners = \corners,
                     -{Latex[scale=3,length=3,width=2]}},
      T/.style args = {#1/#2}{draw, black, rounded corners = \corners,
                     to path={-| node[pos=#1,text=black] {#2}
                     (\tikztotarget)},
                     -{Latex[scale=3,length=3,width=2]}},
      T/.default = /,
      Ti/.style args = {#1/#2}{draw, black, rounded corners = \corners,
                     to path={-| node[pos=#2,text=black] {#1}
                     (\tikztotarget)},
                     {Latex[scale=3,length=3,width=2]}-},
      Ti/.default = /,
]

\baselineskip=0.3cm
\node (annotation) [manual]              {Manual\\IED\\Annotation};
\node (annotationArtefacts) [manual, left = of annotation] {Manual\\Artefact\\Annotation};
\node (PSG) [manual, left = of annotationArtefacts]                 {Sleep\\Scoring};
\node (empty0) [empty, below = of annotation]                       {};
\node (annotationSplit) [empty, below = of empty0]                  {};
\node (alignment) [process, below = of annotationSplit]             {Alignment};
\node (lp) [process, right = of annotation]                         {LP filter};
\node (chansel) [decision, above = of lp]                           {Channel\\Selection};
\node (DB) [io, above = of chansel]                                 {Offline\\Database};
\node (DA) [startstop, above = of DB]                               {Online\\Data\\Acquisition};
\node (continuousIED) [io] at (alignment -| lp)                     {Continuous\\IED Data};
\node (hp) [process, right = of lp]                                 {HP filter};
\node (empty1) [empty, right = of hp]                               {};
\node (epoching) [process, below = of alignment]                    {Epoching};
\node (windowing) [process] at (epoching -| hp)                     {Windowing};
\node (IED) [io, below = of epoching]                               {Epoched\\Data};
\node (window) [io, below = of windowing]                           {Windowed\\Data};
\node (clusterIED) [process, below = of IED]                        {Clustering};
\node (templateIED) [io, below = of clusterIED]                     {IED\\Template};
\node (search) [decision] at (templateIED -| continuousIED)         {IED\\Template\\Search};
\node (detected) [io, below = of search]                            {Detected\\IED};
\node (continuousSpike) [io, below = of empty1]                     {Continuous\\Spike Data};
\node (prewhiten) [process, right = of continuousSpike]             {Pre-\\whitening};
\node (artefactSpike) [empty] at (annotationSplit -| prewhiten)     {};
\node (threshold) [process, below = of prewhiten]                   {Threshold\\Detection};
\node (spikes) [io, below = of threshold]                           {Spike-\\times};
\node (artifactRejectionSpike) [decision, below = of spikes]        {Artefact\\rejection};
\node (clusterSpike) [process, below = of artifactRejectionSpike]   {Clustering};
\node (templateSpike) [io] at (templateIED -| clusterSpike)         {Spike\\Template};
\node (searchSpike) [decision] at (templateIED -| continuousSpike)  {Spike\\Template\\Search};
\node (spikeCluster) [io, below = of searchSpike]                   {Spike\\Cluster};
\node (selectSpike) [decision, below = of spikeCluster]             {Evaluate\\Cluster};
\node (phy) [manual, right = of selectSpike]                        {Visual\\Evaluation};
\node (SUA) [io, below = of selectSpike]                            {Isolated\\Units};
\node (empty5) [empty, below = of SUA]                              {};
\node (artifactRejectionWindow) [decision] at (SUA -| window) {Artefact\\rejection};
\node (combineIED) [combine, label={above left:{\textit{c}}}] at (empty5 -| detected)      {};
\node (empty4) [empty]  at  (combineIED -| artifactRejectionWindow)   {};
\node (combineWindow) [combine, label={above left:{\textit{d}}}, below = of empty4] {};
\node (outWindow) [io, below = of combineWindow]                         {Window\\$\times$ Stage\\$\times$ Unit};
\node (outIED) [io] at (outWindow -| lp)                            {IED\\$\times$ Stage\\$\times$ Unit};
\node (empty3) [empty, below = of outIED]                           {};
\node (empty4) [empty, below = of outWindow]                        {};
\node (artifactRejectionIED) [decision, below = of detected]        {Artefact\\rejection};

\draw[T](chansel.south) to                                  (annotation);
\draw[T](DB) to                                             (annotationArtefacts);
\draw[T](chansel.west) -| node [pos=0.85, right] {PSG}      (PSG);
\draw[arrow](DB) to                                         (chansel);
\draw[arrow](chansel)  to node [pos=0.10, right] {Macro}    (lp);
\draw[T](chansel.east) -| node [pos=0.85, right] {Micro}    (hp);
\draw[arrow](DA) to                                         (DB);
\draw[arrow](continuousIED) to                              (alignment);
\draw[T](continuousIED.east) to                             (windowing.north);
\draw[arrow](alignment) to                                  (epoching);
\draw[arrow](windowing) to                                  (window);
\draw[arrow](epoching) to                                   (IED);
\draw[arrow](IED) to                                        (clusterIED);
\draw[arrow](window) to                                     (artifactRejectionWindow);
\draw[arrow](clusterIED) to                                 (templateIED);
\draw[arrow](templateIED) to                                (search);
\draw[arrow](continuousIED) to                              (search);
\draw[arrow](search) to                                     (detected);
\draw[arrow](detected) to                                   (artifactRejectionIED);
\draw[T](hp.east) to                                        (continuousSpike.north);
\draw[arrow](continuousSpike) to                            (prewhiten);
\draw[arrow](prewhiten) to                                  (threshold);
\draw[arrow](threshold) to                                  (spikes);
\draw[arrow](artifactRejectionSpike) to                     (clusterSpike);
\draw[arrow](spikes) to                                     (artifactRejectionSpike);
\draw[arrow](templateSpike) to                              (searchSpike);
\draw[arrow](searchSpike) to                                (spikeCluster);
\draw[arrow](spikeCluster) to                               (selectSpike);
\draw[T](spikeCluster.east) to                              (phy.north);
\draw[arrow](clusterSpike) to                               (templateSpike);
\draw[arrow](selectSpike) to                                (SUA);
\draw[arrow](phy) to                                        (selectSpike);
\draw[arrow](combineWindow) to                              (outWindow);
\draw[Ti](combineWindow.east) to                            (SUA.280);
\draw[L] (annotationArtefacts.270) to                       (artifactRejectionIED -| annotationArtefacts.270) to (artifactRejectionIED);
\draw[T, name path=line 5](spikes) to                       (searchSpike.west);
\draw[arrow](outIED) to                                     (empty3);
\draw[arrow](outWindow) to                                  (empty4);

%% first times don't draw the long one because it has to jump as well
\JumpOver{(annotation) to (alignment)}[L, draw=none]{(annotationArtefacts.290) to (threshold -| annotationArtefacts.290) to (threshold -| continuousSpike.290) to (artifactRejectionSpike -| continuousSpike.290) to (artifactRejectionSpike.west)};

\JumpOver{(lp) to (continuousIED)}[L, draw=none]{(annotationArtefacts.290) to (threshold -| annotationArtefacts.290) to (threshold -| continuousSpike.290) to (artifactRejectionSpike -| continuousSpike.290) to (artifactRejectionSpike.west)};

\JumpOver{(continuousSpike) to (searchSpike)}[L, draw=none]{(annotationArtefacts.290) to (threshold -| annotationArtefacts.290) to (threshold -| continuousSpike.290) to (artifactRejectionSpike -| continuousSpike.290) to (artifactRejectionSpike.west) to (spikes) to (continuousIED)};

%% now draw and jump
\JumpOver{(annotationArtefacts.290) to (threshold -| annotationArtefacts.290) to (threshold -| continuousSpike.290) to (artifactRejectionSpike -| continuousSpike.290) to (artifactRejectionSpike.west)}[T, draw=none]{(spikes) to (searchSpike.west)};

%% south part of schema

\JumpOver{(artifactRejectionIED) to (combineIED)}[L]{(annotationArtefacts.250) to (artifactRejectionWindow -| annotationArtefacts.250) to (artifactRejectionWindow)};
\draw[Ti](combineIED.west) to (PSG.280);
\JumpOver{(combineIED) to (outIED)}[Ti]{(combineWindow.west) to (PSG.260)};
\JumpOver{(artifactRejectionWindow) to (combineWindow)}[Ti]{(combineIED.east) to (SUA.260)};

\end{tikzpicture}
}

